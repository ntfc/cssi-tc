\chapter{Grupo IV}
\section{Implementação}
A implementação do CBC-MAC em \sage\ é bastante simples. Usa-se como função pseudo-aleatória a cifra por blocos AES, através da biblioteca \verb|pycrypto|. Vamos usar como valores \textit{default} chaves de $n = 128$ bits e $\ell(n) = 5$, ou seja, mensagens de comprimento $640$ bits ($80$ caractéres) divididas em blocos de $128$ bits ($16$ caractéres). Considerem-se a mensagem $m = m_1\ ||\ m_2\ ||\ \cdots\ ||\ m_{\ell(n)}$, com as \textit{tags} $t = t_1\ ||\ t_2\ ||\ \cdots\ ||\ t_{\ell(n)}$ ($\mathsf{Mac_k}(m)$ apenas retorna o bloco $t_{\ell(n)}$).
%%%%% Return all blocks
\section{Retornar todos os blocos}
Retornar todos os blocos da \textit{tag} torna o CBC-MAC inseguro pois torna-se bastante fácil falsificar MAC's. Na imagem~\ref{img:cbcmac} é demonstrado o mecanismo de falsificação de MAC's para uma mensagem em que é criada uma mensagem falsa $m' =  (t_1 \oplus m_2)\ ||\ m_3\ ||\ \cdots\ ||\ m_\ell(n)\ ||\ (t_{\ell(n)} \oplus m_1)$. A \textit{tag} usada para verificação é apenas $t_{\ell(n)}$.
\begin{figure}[h!]
\centering
  \tikzset{line/.style={draw, -latex',shorten <=1bp,shorten >=1bp}}

\begin{tikzpicture}
  %\draw[help lines] (0,0) grid (9, 9);
  \foreach \i in {0,...,2} { %%% CHANGE NUMBER OF BLOCKS HERE! %%%
    \pgfmathsetmacro{\z}{3.5*\i}
    % draw k
    \node (k\i) at (-0.5 + \z, 2) {\textsf{k}};
    % draw box
    \path[draw] (0.5 + \z, 1.5) -- ++(1.5, 0) -- ++(0, 1) -- ++(-1.5, 0) -- ++(0, -1);
    % draw text
    \node (F\i) at (1.25 + \z, 2) {$\mathsf{F_k}$};
    % this macros cast the float to integer
    \pgfmathtruncatemacro{\t}{1+\i}
    \pgfmathtruncatemacro{\m}{2+\i}
    \pgfmathtruncatemacro{\r}{4 - \i} %%% CHANGE NUMBER OF BLOCKS HERE! %%%
    %%%% if \i != 0 %%%%
    \ifthenelse{\NOT 0 = \i} {
      % draw XOR gate in all blocks, except the first
      \node (XOR\i)[XOR] at (1.25 + \z, 3.75) {};
      % draw arrows above F_k
      \path[line] (XOR\i) edge ++(0, -1.25) % xor to F_k
                  (k\i) edge ++(1, 0); % k to F_k
      %%%% last block %%%%
      \ifthenelse{2 = \i}{ %%% CHANGE NUMBER OF BLOCKS HERE! %%%
        % draw input block
        \node (m\i) at (1.25 + \z, 5) {$t_{\ell(n)} \oplus m_{1}$};
        % draw output block
        \node (t\i) at (1.25 + \z, 0) {$t_{\ell(n)}$};
      }
      %%%% all blocks, except first and last %%%%
      {
        % draw input block
        \node (m\i) at (1.25 + \z, 5) {$m_{\r}$};
        % draw output block
        \node (t\i) at (1.25 + \z, 0) {$t_{\m}$};
      }
      % draw arrows above F_k
      \path[line] (m\i) edge (XOR\i); % input block to xor
    }
    %%%% first block %%%%
    {
      % draw input block
      \node (m\i) at (1.25 + \z, 5) {$t_{\t} \oplus m_{\m}$};
      % draw arrows above F_k
      \path[line] (m\i) edge ++(0, -2.5) % input block to F_k
                  (k\i) edge ++(1, 0); % k to F_k
      % draw output block
      \node (t\i) at (1.25 + \z, 0) {$t_{\m}$};
      
    }
    % dont add padding to the beggining of the line
    %\tikzset{line/.style={draw, -latex',shorten <=0bp,shorten >=1bp}}
    \path[line] (1.25 + \z, 1.5) -- (t\i); % F_k box to block
    %\tikzset{line/.style={draw, -latex',shorten <=1bp,shorten >=1bp}}
  }
  
  %%%% draw only arrows
  \foreach \i in {0,...,2} { %%% CHANGE NUMBER OF BLOCKS HERE! %%%
    \pgfmathsetmacro{\z}{3.5*\i}
    \pgfmathtruncatemacro{\t}{1+\i}
    \pgfmathtruncatemacro{\u}{1-\i}
    \ifthenelse{\NOT 2 = \i} { %%% CHANGE NUMBER OF BLOCKS HERE! %%%
      \ifthenelse{1 = \i}{  %%% CHANGE NUMBER OF BLOCKS HERE! %%%
        \node (dots1) at (3 + \z, 5) {$\cdots$};
        \node (dots1) at (3 + \z, 3.25) {$\cdots$};
        \node (dots1) at (3 + \z, 0) {$\cdots$};
        \path[line] (4 + \z, 3.75) -- (XOR\t);
      }
      {
        % draw arrow from F_k to XOR
        \path[line] (1.25 + \z, 1) -- ++(1.25, 0) -- ++(0, 2.75) -- (XOR\t);
      }
    }
    {
    }
    
  }

  % dont add padding to the beggining of the line
  \tikzset{line/.style={draw, -latex',shorten <=0bp,shorten >=1bp}}
  %\path[line] (3, 4) -- (t1XORm3);
\end{tikzpicture}
  \caption{Falsificação de MAC's quando todos os blocos da \textit{tag} são retornados por \textsf{Mac}.}
  \label{img:cbcmac}
\end{figure}
%
%%%%%% Use random IV
\section{Usar um \textsf{IV} aleatório}
Usando um \textsf{IV} aleatório $iv$ ainda se torna mais fácil falsificar mensagens (sendo que \textsf{Mac} apenas retorna o último bloco como \textit{tag}). Basta para isso criar a nova mensagem $m' = (m_1 \oplus iv)\ ||\ m_2\ ||\ \cdots\ ||\ m_{\ell(n)}$ e a tag $t' = t$ tal que $\mathsf{Vrfy_k}(m', t') = \top$.
\begin{figure}[h!]
\centering
  \input{img/randomiv}
  \caption{Falsificação de MAC's quando se utiliza um \textsf{IV} aleatório.}
  \label{img:randiv}
\end{figure}
%%%% exemplo ataque
\section{Implementação e execução \sage}
Depois de implementado o CBC-MAC, criaram-se dois métodos na classe \verb|CBCMAC| para falsificar mensagens:
\begin{description}
  \item[ForgeMacRandomIV(msg, tag, iv)] Retorna o par $(msg', tag)$
  \item[ForgeMacAllTags(msg, tag)] Retorna o par $(msg', tag')$
\end{description}
Além disso, o método \textsf{Mac} tem uma variante insegura, bastante para isso incluir o argumento \verb|secure=False| na sua invocação.\\
Segue-se o exemplo de uma sessão \sage\ em que se falsificam algumas mensagens.
\begin{lstlisting}[style=Bash,caption=Exemplo de falsificação de MACs em \sage]
sage: CBC = CBCMAC(128,5)
sage: key = MyKey(length=128)
sage: m = 'Isto e o CBC-MAC. Vamos forgar esta mensagem usando dois metodos. Wish me luck!!'
sage: iv = randomIV()
sage: t = CBC.Mac(key, m, iv=iv) # auth msg using random IV
sage: (m2, t2) = CBC.ForgeMacRandomIV(m, t, iv) # forge new msg and tag
sage: CBC.Vrfy(key, m, t, iv) # Vrfy using the random IV
True
sage: m2 ; CBC.Vrfy(key, m2, t2)
'<\xd0w\xc2#\xc6\xdaa[\x8eH\x94\x0el0\xb2. Vamos forgar esta mensagem usando dois metodos. Wish me luck!!'
True
sage: CBC.Vrfy(key, m, t) # Vrfy using IV = 0
False
sage: (iv,t) = CBC.Mac(key, m, secure=False) # auth msg and return all blocks
sage: (m3, t3) = CBC.ForgeMacAllTags(m, t) # forge new msg and tag
sage: m3 ; CBC.Vrfy(key, m3, t3) # Vrfy new forged msg
'\xad\x11q\xe0X IG\xf8\xd4\x1c\x10\x10\x85\xfaMsta mensagem usando dois metodos. Wish me luck!!\x0f\xdd0\x9fh=I\xdbB=\x8f"g\x88\x00\x96'
True
\end{lstlisting}
%\begin{btSect}{bib/bibz04}
% \section{Bibliografia}
 %\btPrintCited
 %\btPrintNotCited
% \btPrintAll
%\end{btSect}

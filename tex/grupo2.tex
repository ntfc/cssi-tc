\chapter{Grupo II}
O One-Time Pad é uma cifra que consiste em fazer o XOR do texto limpo com uma chave aleatória do mesmo tamanho que a mensagem a transmitir. Esta cifra, no caso de utilizar uma chave verdadeiramente aleatória e esta se mantiver em segredo torna o criptograma indecifrável, garantido desta forma segurança perfeita. Contudo, no caso de uma chave ser usada mais do que um vez, essa segurança perfeita deixa de existir e a mensagem pode então ser decifrada por um atacante.\\
Assim, se $C_1 = m_1 \oplus k$ e $C_2 = m_2 \oplus k$ e for feita a operação $C_3 = C_1 \oplus C_2$ temos $C_3 = m_1 \oplus m_2$. Desta forma realizamos esta operação para todos os pares possíveis de criptogramas dados. De assinalar que a referida operação de xor neste exercício corresponde à função decifrar.\\
Posto isto, temos um conjunto de novos criptogramas em que um deles deixa de ter uma distribuição de caracteres verdadeiramente aleatória, o que permite que através algumas técnicas usadas em criptoanálise seja possível identificar os criptogramas com maior probabilidade de corresponderem ao que se pretende encontrar. As técnicas que optámos por utilizar foram duas já referidas no guião anterior: Qui-quadrado e o Índice de Coincidência. Os resultados obtidos não foram tal como se esperava muito claros, pois trata-se de um xor entre dois textos limpos apenas composto por letras do alfabeto.\\
Análisamos os resultados obtidos para os 15 criptogramas que apresentam maior probabilidade nas duas técnicas de corresponderem ao criptograma que se pretende descobrir e apenas um é comum a ambas. O criptograma em questão corresponde ao xor dos criptogramas 6 e 18 e apresenta o valor de 878.14 na técnica do Qui-quadrado e de 0.041 relativamente ao Índice de Coincidência. Podemos então concluir que estes dois criptogramas foram muito provavelmente cifrados com a mesma chave. \newline \newline
Em relação ao segundo desafio, optámos, em primeiro lugar, por seguir as mesmas técnicas utilizados no desafio anterior. Contudo verificamos que todos os novos criptogramas apresentam valores semelhantes para o Índice de Coincidência. Posto isto, decidimos analisar a questão da frequência de ocorrência das letras do alfabeto nos criptogramas de forma a podermos encontrar indícios de qual será o criptograma que pretendemos identificar.\\ 
Depois de nos focarmos sobretudo na ocorrência média de cada letra nos criptogramas chegamos a uma conclusão que consideramos poder indicar o criptograma que representa o xor de dois textos limpos.  A nossa teoria baseia-se em fazer a média de ocorrência de cada uma das letras do alfabeto no conjunto dos criptogramas, de seguida para cada criptograma é calculada a diferença de ocorrência de cada letra relativamente à média, a soma das diferenças de cada letra do alfabeto num criptograma dá-nos o que denominamos por desvio.  Esta teoria baseia-se no facto de todos os criptogramas, exceptuando o que é composto pelo xor de dois textos limpos, apresentarem uma distribuição aleatória de letras, assim sendo o criptograma a encontrar será um dos que apresentará um desvio maior relativamente à média, pelo facto de não ter uma distribuição verdadeiramente aleatória. \\
Optámos então por utilizar esta técnica em conjunto com o Qui-quadrado e comparar os resultados obtidos. Assim, tal como no primeiro desafio,  comparando os 15 resultados que apresentam maior probabilidade em cada uma das técnicas, alcançamos 3 resultados em comum: o xor dos criptogramas 5 e 7, 1 e 18 e por fim 17 e 18. Acreditamos então que os criptogramas cifrados com a mesma chave são muito provavelmente um destes três pares. 
